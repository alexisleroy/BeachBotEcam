

%gestion
\section{Méthode SCRUM}
%~~~~~~~~~~~~~~~~~~~~~~~~~~~~~~~~~~~~~~~~~~~~~~~~~~~~~~~~~~~~
\subsection{Organisation}
\paragraph{}
Dans l'optique du projet robot, nous avons suivi une méthode agile, SCRUM, qui correspondait à nos besoins d'organisation. En effet, pour un travail en comité réduit avec une échéance suffisamment longue, cette méthode offre un bon système de management des troupes et de gestion de l'évolution du projet.

\noindent Arnaud fut nommé responsable du groupe, Alexis responsable de la plateforme d'échange git et José responsable de l'impression 3D lors de la première séance. Le reste des responsabilités (technique, graphique, recherche) dépend des délivrables et de l'organisation du travail. On peut donc dire qu'aucun poste n'est réellement définitif.

\paragraph{}
L'avancement du projet est rythmé par 2 types d'incrémentation : 
\begin{description}
	\item[Ticket] Objectif simple à atteindre pour la séance suivante (1-2 semaines) ou à moyen-terme;
	\item[Délivrables] Objectif plus important sur une durée d'environ 1 mois.
\end{description}

\noindent En suivant cette technique, nous pouvons facilement constater si le projet avance comme on le souhaite ou si du retard s'accumule et donc s'adapter si nécessaire.

\paragraph{}
Chaque séance de travail commence par une stand up meeting dirigé, dans la mesure du possible, par un animateur différent à chaque fois.

\noindent Celui-ci commence par vérifier, en passant de membre en membre, ce qui a été réalisé (a-t-il terminé son délivrable du jour?), ce qui doit encore être effectué et les obstacles rencontrés.

\noindent Ensuite, après avoir contrôler les objectifs atteints/non atteints, il attribue les prochains délivrables aux équipiers les plus aptes et volontaires. En général, nous travaillons par équipe de 1 à 3 en fonction de l'importance du délivrable. 

\noindent En plus de cette partie d'assignation de tâche, cette séance est également ponctuée de prise d'opinions pour mettre en place ou corriger le planning, ajouter de nouveaux délivrables et lancer des pistes/conseils à propos de la réalisation de certains objectifs en cours.

\subsection{Délivrables}

1\up{er} délivrable : 19/11.
\begin{itemize}
	\item Table fonctionnelle;
	\item Schéma bloc;
	\item Architecture.
\end{itemize}

\noindent Ce premier délivrable comprend la partie conception de la table qui se divise en plusieurs étapes allant de l'acquisition du bois à la peinture. Toute l'équipe a dû se relayer pour avoir la table achevée pour le jour désigné. La partie schéma bloc et architecture a abouti dans un prototype fonctionnel, mais risque fortement d'évoluer dans les prochains mois en fonction des ajouts et problèmes à venir.

\paragraph{}
2\up{ème} délivrable : 10/12.
\begin{itemize}
	\item Plan du châssis;
	\item Nomenclature à compléter;
	\item Commande du 2\up{ème} robot et des autres composants (capteurs, câbles, ...);
	\item Régulation des moteurs.
\end{itemize}

\noindent Dans ce cas-ci, la création de la nomenclature a permis d'effectuer les commandes qui seront réceptionnées pour janvier au plus tard. Le plan de châssis fut retardé à cause de l'achat des moteurs. Au niveau de la régulation, notre système est fonctionnel en théorie. Il doit encore être testé sur le terrain, mais, pour cela, il est nécessaire d'avoir un prototype de robot opérationnel.

\subsection{Conclusion}

Jusqu'à maintenant, cette méthode porte ses fruits. En effet, tous les délivrables mis en place ont été remplis correctement hormis le plan du châssis qui requiert encore quelques retouches. Au total, ce sont 33 tickets sur 38 qui furent réalisés. Les 5 non-terminés représentent, des objectifs trouvés en cours de route, mais qui seront à réaliser plus tard\footnote{Non prioritaire et sans date assignée.}. On peut citer par exemple la mise en place du line tracking ou la détection d'un objet avec la caméra via OpenCV.

\noindent On peut tout de même signaler que, dans un premier temps, nous eûmes quelques soucis à établir des tickets simples et concis causant des problèmes d'organisation lors des 2 premières séances. Heureusement, avec l'expérience, nous avons pu adapter notre méthode en divisant certains tickets en plusieurs. Par exemple, un ticket "Réaliser la table" s'est vu diviser en 5 tickets : 
\begin{itemize}
	\item "Commande du bois";
	\item "Découpe et livraison";
	\item "Assemblage";
	\item "Peinture";
	\item "Tracés du logo et de la ligne noire".
\end{itemize}

\section{GIT}
%~~~~~~~~~~~~~~~~~~~~~~~~~~~~~~~~~~~~~~~~~~~~~~~~~~~~~~~~~~~~
\paragraph{}
L'utilisation de la plateforme d'échange GIT nous a permis de travailler de façon synchronisée sans empiéter sur le travail d'autrui. Effectivement, la possibilité de récupérer, de consulter, de modifier des éléments (fichiers, ligne de code,... ) du projet tout en gardant un historique des changements effectués offre une grande liberté dans l'organisation du travail (en laboratoire, à domicile,... ).

\paragraph{}
Le seul bémol fut, pour certains, l'adaptation au fonctionnement \textbf{peu user-friendly} de GIT. Cependant, avec un peu de patience et d'expérience, toute l'équipe a pu constater l'efficacité de cette plateforme dans ce genre de projet qui la rend incontournable.

